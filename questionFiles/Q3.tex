%%%%%%%%%%%%%%%PART-A(Question#03--
%%%%%%%%%%%%%%%%%%%%%%%%%%%%%%%%%%%%%%%%%%%%%%%%%%%%%%%%%%%%%%%%%%%%%%%%%%%%%%%%%%%%%%%%%%%%%%%%%%%%%	
%%%%%%%%%%%%%%%%%%%%%%%%%%%%%%%%%%%%%%%%%%%%%%%%%%%%%%%%%%%%%%%%%%%%%%%%%%%%%%%%%%%%%
\question 
\begin{parts}
%%%%%%%%%%%%%%%%%%%%%%%%%%%%%%%%%%%%%%%%%%%%%%%%%%%%%%%%%%%%%%%		
\part[4]\color{red} If you want to place two figures side by side then use minipage environment. As for example: \color{black}For the circuit shown in Fig.~\ref{fig:ckt-thevenin-dc}, calculate the current in the $10~\unit{\ohm}$ resistance. Use Thevenin's theorem only.
		
			

\begin{figure}[H]
\centering
% First figure
\begin{minipage}[t]{0.48\textwidth}
\centering
\begin{tikzpicture}[scale=0.7]

% Circuit elements
\draw (3, 4) to[american resistor, l={$8~\Omega$}] (6, 4);
\draw (3, 4) to[american resistor, l={$2~\Omega$}] (3, 0);
\draw (9, 4) to[american resistor, l_={$10~\Omega$}] (9, 0);
\draw (1, 4) to[american resistor, l_={$5~\Omega$}] (1, 2);
\draw (1, 2) to[battery1, l_={$20~V$}] (1, 0);
\draw (6, 4) to[battery, l_={$12~V$}] (9, 4);
\draw (1, 0) -- (9, 0);
\draw (1, 4) -- (3, 4);
\draw node[ocirc] (N1) at (9, 4) {} node[anchor=south] at (N1.north){$A$};
\draw node[ocirc] at (9, 0) {};
\draw node[ocirc] (N2) at (9, 0) {} node[anchor=north] at (N2.south){$B$};

\end{tikzpicture}
\caption{Circuit for Thevenin's Theorem}
\label{fig:ckt-thevenin-dc}
\end{minipage}
%%%%%%%%%%%%%%%%%%%%%%%%%%%%%%%%%%%%%%%%%%%%%%%%%%%%%%%%%%%%%%%%%%%
\hfill
%%%%%%%%%%%%%%%%%%%%%%%%%%%%%%%%%%%%%%%%%%%%%%%%%%%%%%%%%%%%%%%%%%%
% Second figure
\begin{minipage}[t]{0.48\textwidth}
\centering
\begin{tikzpicture}[scale=0.7]

\draw (4, 5) to[american resistor, l_={$3~\Omega$}] (4, 3);
\draw (4, 3) to[american resistor, l_={$5~\Omega$}] (4, 1);
\draw (4, 3) to[american resistor, l={$1~\Omega$}] (6, 3);
\draw (9, 3) to[american resistor, l={$4~\Omega$}] (9, 1);
\draw (4, 5) to[american resistor, l={$2~\Omega$}] (9, 5);
\draw (2, 1) to[american current source, l={$4~A$}] (2, 5);
\draw (6, 3) to[american controlled voltage source, l={$5i_o$}] (9, 3);
\draw (4, 1) to[american voltage source, l_={$20~V$}] (9, 1);
\draw (2, 5) -- (4, 5);
\draw (2, 1) -- (4, 1);
\draw (9, 5) -| (9, 3);

% Current label i_o
\node at (4.7, 2) {$i_o$};
\draw[->] (4.5, 2.5) -- (4.5, 1.5);

\end{tikzpicture}
\caption{Circuit for Superposition Theorem}
\label{fig:ckt-superposition-dc}
\end{minipage}
\end{figure}

\begin{solution}

\end{solution}
%%%%%%%%%%%%%%%%%%%%%%%%%%%%%%%%%%%%%%%%%%%%%%%%%%%%%%%%%%%%%%%	
\part[4]What is super position theorem? Find $i_o$ in the circuit in Fig.~\ref{fig:ckt-superposition-dc} using the superposition theorem. 
		
\begin{solution}

\end{solution}
%%%%%%%%%%%%%%%%%%%%%%%%%%%%%%%%%%%%%%%%%%%%%%%%%%%%%%%%%%%%%%%
\part[2] \color{red}If a table is needed to insert, then you can use either tabular or tabulrx. In the table as shown in Table~\ref{table:someItems}, the data are given for a general purpose Silicon diode, draw the I-V characteristic curve. 

\color{black}
\begin{table}[H]
\centering
\caption{Your Table Title Here}
\label{table:someItems}
\begin{tabularx}{0.8\textwidth} { 
  | >{\raggedright\arraybackslash}X 
  | >{\centering\arraybackslash}X 
  | >{\raggedleft\arraybackslash}X | }
 \hline
 SL.no & Forward bias voltage (V) & Forward bias current(mA) \\
 \hline
  1 & 0 & 0 \\
 \hline
  2 & 0.2 & 0.0 \\
 \hline
  3 & 0.4 & 0.1 \\
 \hline
  4 & 0.5 & 0.5 \\
 \hline
  5 & 0.53 & 1.0 \\
 \hline
  6 & 0.6 & 8.2 \\
 \hline
  7 & 0.66 & 19.5 \\
 \hline
  8 & 0.7 & 53.5 \\
 \hline
  9 & 0.71 & 83.1 \\
 \hline
  10 & 0.73 & 112.7 \\
 \hline
 
\end{tabularx}
\end{table}

\begin{solution}

\end{solution}
%%%%%%%%%%%%%%%%%%%%%%%%%%%%%%%%%%%%%%%%%%%%%%%%%%%%%%%%%%%%%%%
\end{parts}


