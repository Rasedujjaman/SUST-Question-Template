%%%%%%%%%%%%%%%%%%%%%%%%%%%%%%%%%%%%%%%%%%%%%%%%%%%%%%%%%%
%%%%%%%%%%%%%%%%%%%%%%  Question#01  %%%%%%%%%%%%%%%%%%%%%
%%%%%%%%%%%%%%%%%%%%%%%%%%%%%%%%%%%%%%%%%%%%%%%%%%%%%%%%%%
\begin{comment}
  Put all the topics for this question.
\end{comment}
%%%%%%%%%%%%%%%%%%%%%%%%%%%%%%%%%%%%%%%%%%%%%%%%%%%%%%%%%%
\question 
\begin{parts}
%%%%%%%%%%%%%%%%%%%%%%%%%%%%%%%%%%%%%%%%%%%%%%%%%%%%%%%%%%
\part[2] What are luminescent and photo-luminescent processes?


\begin{solution}

\end{solution}
%%%%%%%%%%%%%%%%%%%%%%%%%%%%%%%%%%%%%%%%%%%%%%%%%%%%%%%%%%
\part[2] How is the concept of quasi-Fermi level useful in semiconductors?


\begin{solution}

\end{solution}
%%%%%%%%%%%%%%%%%%%%%%%%%%%%%%%%%%%%%%%%%%%%%%%%%%%%%%%%%%
\bdpart[2+2] \color{red} If you want the marks to be displayed as \(2+2\), use the command \verb|\bdpart[2+2]| instead of the regular \verb|\part[]| command. \color{blue} As for example: \color{black} What is molecular orbital? Explain bonding and anti-bonding in molecular orbital.

\begin{solution}

\end{solution}
%%%%%%%%%%%%%%%%%%%%%%%%%%%%%%%%%%%%%%%%%%%%%%%%%%%%%%%%%% 
\bdpart[2+2] \color{red} If you want the subparts of the question to be appeared inline the use the command \verb|\begin{inlinesubparts} \item, \item, etc.---\end{inlinesubparts}|. \color{blue} As for example: \color{black} Draw schematic diagrams to show splitting of two molecular orbitals due to resonance 
interactions with the orbitals:
\begin{inlinesubparts}
	\item \color{red}having same energy
	\item \color{red}having different energies.\color{black}
\end{inlinesubparts}
 

	
\begin{solution}

\end{solution}
%%%%%%%%%%%%%%%%%%%%%%%%%%%%%%%%%%%%%%%%%%%%%%%%%%%%%%%%%%
\end{parts}
