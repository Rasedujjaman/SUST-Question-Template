%%%%%%%%%%%%%%%%%%%%%%%%%%%%%%%%%%%%%%%%%%%%%%%%%%%%%%%%%%
%%%%%%%%%%%%%%%%%%%%%%  Question#05  %%%%%%%%%%%%%%%%%%%%%
%%%%%%%%%%%%%%%%%%%%%%%%%%%%%%%%%%%%%%%%%%%%%%%%%%%%%%%%%%
\begin{comment}
  Put all the topics for this question.
\end{comment}
%%%%%%%%%%%%%%%%%%%%%%%%%%%%%%%%%%%%%%%%%%%%%%%%%%%%%%%%%%
\question 
\begin{parts}
%%%%%%%%%%%%%%%%%%%%%%%%%%%%%%%%%%%%%%%%%%%%%%%%%%%%%%%%%%
\part[2] \color{red} Using the command \verb|\ce{..}| you can typeset the chemical formula and equations. \color{blue} As for example: \color{black}Write down whether the given chemical formulas are organic or inorganic.

\begin{itemize}
	\item Water: \verb|\ce{H2O}|, \ce{H2O}
	\item Benzene: \verb|\ce{C6H6}|, \ce{C6H6}
	\item Hydrogen peroxide: \verb|\ce{H2O2}|, \ce{H2O2}
	\item Acetic acid: \verb|\ce{C2H4O2}|, \ce{C2H4O2}
	\item Glucose: \verb|\ce{C6H12O6}|, \ce{C6H12O6}
\end{itemize}

\begin{solution}

\end{solution}
%%%%%%%%%%%%%%%%%%%%%%%%%%%%%%%%%%%%%%%%%%%%%%%%%%%%%%%%%% 
\bdpart[2+2] \color{red}The chemical equation can also be written using the command \verb|\ce{}|. \color{blue}As for example: \color{black} Explain the following two chemical equations.

\begin{itemize}
\item \verb|\ce{2H2 + O2 -> 2H2O}| typesets \ce{2H2 + O2 -> 2H2O}
\item \verb|\ce{CO2 + C -> 2 CO}| typesets \ce{CO2 + C -> 2 CO}
\end{itemize}


\begin{solution}

\end{solution}
%%%%%%%%%%%%%%%%%%%%%%%%%%%%%%%%%%%%%%%%%%%%%%%%%%%%%%%%%%
\bdpart[1+4]\color{red} Drawing a molecule consists mainly of connecting groups of atoms with lines. Simple linear formulae can be easily drawn using the chemfig package and using the command \verb|\chemfig{*6((=O)-N(-)-(*5(-N=-N(-)-))=-(=O)-N(-)-)}|, \color{blue}as shown in the following example: \color{black} Identify the given chemical formula.
\[
\chemfig{*6((=O)-N(-)-(*5(-N=-N(-)-))=-(=O)-N(-)-)}
\]


\begin{solution}

\end{solution}
%%%%%%%%%%%%%%%%%%%%%%%%%%%%%%%%%%%%%%%%%%%%%%%%%%%%%%%%%%

\end{parts}
